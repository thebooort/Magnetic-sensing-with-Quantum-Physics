\documentclass[1p]{elsarticle}

\usepackage{lineno,hyperref}
\modulolinenumbers[5]
\usepackage[utf8]{inputenc}
\usepackage[spanish]{babel}
\usepackage{amsmath}
\usepackage{todonotes}
\usepackage{listings}
\usepackage{graphicx}
\usepackage{amsfonts}
\usepackage{wrapfig}
\usepackage[toc,page]{appendix}
\usepackage{amssymb}
\newtheorem{thm}{Teorema}
\newtheorem{lem}[thm]{Lema}
\newdefinition{rmk}{Remark}
\newproof{pf}{Demostración}
\newproof{pot}{Demostración del Teorema \ref{thm2}}
 \graphicspath{ {./images/} }
 \usepackage{listings}
 \usepackage{color}
 
 \definecolor{codegreen}{rgb}{0,0.6,0}
 \definecolor{codegray}{rgb}{0.5,0.5,0.5}
 \definecolor{codepurple}{rgb}{0.58,0,0.82}
 \definecolor{backcolour}{rgb}{0.95,0.95,0.92}
 
 \lstdefinestyle{mystyle}{
 	backgroundcolor=\color{backcolour},   
 	commentstyle=\color{codegreen},
 	keywordstyle=\color{magenta},
 	numberstyle=\tiny\color{white},
 	stringstyle=\color{codepurple},
 	basicstyle=\footnotesize,
 	breakatwhitespace=false,         
 	breaklines=true,                 
 	captionpos=b,                    
 	keepspaces=true,                 
 	numbers=left,                    
 	numbersep=5pt,                  
 	showspaces=false,                
 	showstringspaces=false,
 	showtabs=false,                  
 	tabsize=2
 }
 
 \lstset{style=mystyle}
%%\bibliographystyle{IEEEannot}

%% `Elsevier LaTeX' style
\bibliographystyle{elsarticle-num}
%%%%%%%%%%%%%%%%%%%%%%%
\usepackage{setspace}  
\begin{document}

\begin{frontmatter}

\title{State of art in the Quantum Physics' explanation of the animals' magnetic sense }

%% Group authors per affiliation:
\author{Bartolomé Ortiz Viso}
\address{Master en Física y Matemáticas\\ Universidad de Granada\\23/06/2018}

\begin{abstract}
This work offers a brief look in the quantum physics' explanation of the animals' magnetics sense. It is based on the talks given by Thorsten Ritz in BIOMAT2018 congress, whose main topic was quantum biology. The aim of these pages is to explain the main results in this particular topic (magnetic sensing), its connections with quantum physics and also to offer some other highlights of the talks. Moreover the reader can find some personal opinions and possibles advances that I discussed with Thorsten himself. 
\end{abstract}

\begin{keyword}
 \texttt{Quantum Physics} \sep \texttt{Mathematics}\sep \texttt{Quantum Biology} \sep \texttt{Magnetic Sensing}
\end{keyword}

\end{frontmatter}
\setlength\parindent{0pt}
\linenumbers


\section{Background biológico}
\spacing{1.5}

Nos encontramos ante un delicado campo de estudio. En esta primera sección vamos a exponer los principales hallazgos en cuanto a el conocimiento de que diversas especies animales poseen la capacidad de sentir campos magnéticos, en particular el campo magnético terrestre.

Destacamos que aun hoy sabemos bastante poco sobre este sentido. Si bien los mecanismos hoy día están investigándose, estamos lejos de comprender este sentido completamente. Factores como los mecanismos fisicos implicados, las moleculas receptoras, la transducción de la señal o el procesado neuronal de la misma, son aun objetivo d intenso debate y estudio.

Aun asi, aunque no sepamos todos los mecanismos involucrados, conocemos la existencia y algunos de los límites de este sentido, gracias a los experimentos que se llevan a cabo con diferentes especies animales. Son las respuesta comportamentales que se observan en los sujetos de los experimentos, las que nos presentan el mayor indicio de que este sentido existe y tiene un alto impacto en su dinámica, aun sin saber como funciona exactamente.

En primer lugar, en estos estudios es habitual comprender la tierra como un barra magnética gigante (gigante no nos debe conducir a error: el campo magnético terrestre es difícil de detectar biológicamente). Y, nos interesamos por campo magnético vectorial. Aunque en cada punto podemos encontrar 2 componentes : horizontal y vertical, solemos medir la componente horizontal, y también es destacable el ángulo de inclinación, como veremos durante los experimentos. 

Uno de los animales más habituales en este tipo de experimentos son los pájaros. Es de sobra conocido que muchas aves tienen pautas de migración muy interesantes y complejas, en las que tener sensibilidad al campo magnético terrestre juega un papel crucial.

 En este área los primeros experimentos fueron gracias al desarrollo de instrumental experimental especifico, puesto que los métodos observacionales se mostraron ineficaces. Como se puede leer en \cite{embudo}, los investigadores desarrollaron un embudo \ref{embudo_1} para percibir la dirección que toman los pájaros durante los experimentos. 
  \begin{figure}
  	\centering
  	\includegraphics[width=0.5\textwidth]{emlen_funnel}
  	\caption{Diagrama de embudo de Emlen}
  	\label{embudo_1}
  \end{figure}
  
 Uno de los primeros estudios con esta técnica centrado en el campo magnético terrestre fue llevado a cabo en 1972 \cite{petirrojo}. En el se escogió al petirrojo europeo (\textit{Erithacus rubecula}),el cual está distribuido por toda Europa, principalmente en la región meridional y occidental del continente, donde habita todo el año, siendo migrante parcial en el norte de Europa y noroeste de África.
Se recogieron varios de estos individuos y, una vez dentro de un embudo de Emlen, se procedió a cambiar artificialmente el campo magnético. Se pudieron obtener algunas comclusiones: 
\begin{itemize}
	\item Los pájaros siguieron el norte, tanto en el grupo de control como cuando el norte era alterado artificialmente.\ref{embudo_2}
	\item Los pájaros eran capaces de detectar la inclinación, pero no la polaridad, por lo que la componente vertical es importante.
\end{itemize}
\begin{figure}
	\centering
	\includegraphics[width=0.7\textwidth]{embudopollos}
	\caption{Resultados del cambio en el campo magnetico}
	\label{embudo_2}
\end{figure}

Se compuso así la primera prueba de la sensibilidad magnetica de estos animales. Más tarde fueron incluidos multitud de experimentos relacionados con esta capacidad, los cuales voy a nombrar brevemente para finalizar esta parte.

\begin{itemize}
	\item Se ha observado ciertas relaciones entre la luz y la cantidad de la misma. Aunque no hay una correlacion obvia entre fotoreceptores y comportamiento, sin embargo, la manipulacion con el ciclo regulatorio puede ser la causa. Volveremos sobre este tema durante las evidencia de la importancia de este 
	\item Se ha observado que tapar el ojo derecho produce desorientación, mientras que tapar el ojo izquierdo no, con lo que se plantea la posibilidad de que la "brujula" se encuentre en este ojo.
	\item Se han encontrado evidencias de que los pajaros pueden resetear su percepcion en ciclos de un día. Estudios mostraban que pajaros que tenían una dirección equivocada de manera artificial, corregían su rumbo cuando pasaba un día.
\end{itemize}

Por último, estudios de comportamientos se han llevado a cabo en muchos otros animales, de los cuales, destacamos: 
\begin{itemize}
	\item Gallina común: En este caso el estudio presenta una clara muestra de condicionamiento magnético
	\item Tortuga boba: este animal ofrece unos comportamientos migratorios fascinantes. En experimentos llevados a cabo en piscinas, se pudo observar esa percepción al campo magnético como la mostrada en pájaros.\cite{lohmann1991magnetic}
	\item Las moscas de la fruta también presentan este tipo de comportamiento \cite{cromomosca}
	\item Muchos otros animales: invertebrados, mamíferos, peces, etc.\cite{palomas,bee,anderson1993magnetic,vacha2009radio}
\end{itemize}

Hasta la fecha no se ha encontrado evidencia alguna de que los seres humanos poseamos esta capacidad.

Una vez que tenemos una visión global de este fenómeno, vamos a adentrarnos en los mecanismos que pueden motivarlo.


\section{Mecanismo de par radical}
\spacing{1.5}
Es usual encontrar dos vertientes principales sobre los mecanismos implicados en la magnetorecepcion de los animales.
Por un lado tenemos las particulas de oxido de hierro, sin embargo, estas no tienes ninguna implicacion relacionada con la mecánica cuántica con lo que durante las charlas nos centramos en la vertiente que si tienes estas implicaciones: el mecanismo de radical par.

Durante los años 70 se descubre que campos magneticos débiles puden afectar a las reacciones quimicas, usando luz para  foto-inducir una transferencia de electrones, dando lugar a un par de moleculas con un par de electrones separados pero entrelzados entre sí.

Ahora bien, los electrones tienen la propiedad del spin: todos los electrones giran en un sentido alrededor de su eje. Este giro genera un campo magnético. Como se puede intuir, el spin puede encontrarse en dos estados, que habitualmente estan relacionados mediante el concepto de enlazamiento cuantico. Sin embargo, al separarse pueden darse dos estados: \textit{triplet} (cuando van en el mismo sentido) y \textit{singlet} (cuando van en sentidos opuestos). Según avanza el tiempo y rotan los electrones, estos pasaran de un estado a otro.

Y este proceso es especialmente importante puesto que indice directamente en la mecánica de las reacciones químicas: existen reacciones químicas que solo se producen en uno de los dos estados. Con lo que este cambio periódico en los spin de los electrones retrasa la reacción química. Gráficamente, se puede entender con el diagrama \ref{diagrama}.


\begin{figure}
	\centering
	\includegraphics[width=0.7\textwidth]{diagrama}
	\caption{Diagrama del proceso radical-pair. De la charla de T. Ritz, basado en \cite{schulten1978semiclassical}}
	\label{diagrama}
\end{figure}


Este proceso no pertence solo a los avances teóricos si no que se han podido observar experimentalmente sus implicaciones, como en \cite{maeda2008chemical}.

¿Cómo relacionamos esto con el sentido magnético? Si los animales, en particular los pájaros, poseen este mecanismo para orientarse, entonces los campos magnéticos de alta frecuencia afectaran a la sincronización del mecanismo de radical par (si el campo es lo suficientemente fuertes y tiene la frecuencia adecuada). 
Además, si realizamos este tipo de experimentos, estos no deberían alterar las partículas de oxido de hierro, con lo que sabremos si el mecanismo de par radical influye directamente sobre los animales. 


\section{Sensores óptimos vía moleculas fotoreceptoras}
\spacing{1.5}
Durante las charlas hablamos sobre la experimentación para conseguir un sensor magnético basado en los avances cuánticos de la manera más optima posible.  

Para ello revisitamos algunos de los estudios que se basan principalmente en el enfoque de los sistemas radical-pair y moléculas fotoreceptoras. En particular, los criptocromos, que son los que han mostrado una posibilidad alta de estar involucrados .Los criptocromos son una clase de fotorreceptores de luz azul de plantas y animales que constituyen una familia de flavoproteínas, pueden encontrarse, por ejemplo, en los ojos de algunos pájaros migratorios.

\begin{figure}

	\includegraphics[width=1\textwidth]{ojos}
	\caption{Mecanismo implicado en el sentido magnetico de los pájaros dentro de sus ojos }
	\medskip
	\footnotesize
	De \cite{procopio2016inhomogeneous}:La retina es una capa sensible a la luz en la parte posterior del ojo que convierte las señales de luz, que entran en el ojo de la lente, en señales electroquímicas y transmite estas señales al cerebro a través del nervio óptico. (b) Sección de la retina que muestra los cinco tipos de células dispuestas en capas. La señal primaria se genera en las células del cono del fotorreceptor, pasa a otras capas de células y se transmite al cerebro por las células ganglionares. (c) Disco de los segmentos externos de una célula de cono donde se sugiere que el foto-magnetoreceptor criptocromo está localizado (d) Sección de una membrana de disco de una celda de cono. Las proteínas de rodopsina, representadas en azul, realizan la foto-transducción primaria de información visual. Las proteínas criptocromo, representadas en verde pueden realizar la transducción primaria de información magnética, a través de la formación de reacciones de par radical magnéticamente sensibles.
	\label{pollos_1}
\end{figure}

Se puede ver en \ref{pollos_1} ejemplificado el proceso que siguen estos animales para procesar las señales visuales. También extraemos de \cite{cromogeneral} la imagen \ref{criptos_1} donde se muestran varios mecanismos de par radical observados en otras especies.

\begin{figure}
	
	\includegraphics[width=1\textwidth]{diferentes_animales_cripto}
	\caption{Representación simplificada de tres osciladores moleculares, basados en su composición de criptocromo}
	\footnotesize
	Las líneas rojas indican inhibición, y las flechas discontinuas negras indican la degradación inducida por la luz. Destacamos de las abreviaturas: C, criptocromo
	\label{visual}
\end{figure}




Con esta idea en mente se llevan a cabo multitud de experimentos donde se refleja la importancia de la luz en la magnetopercepcion. Dentro de ellos no interesamos en aquellos que nos arrojen luz sobre como perciben los animales estas características con el fin de, no sólo entender estos procesos, si no tambien para, en un futuro, poder diseñar dispositivos similares. 

Actualmente la existencias de evidencias que apoyan estas teorías es innegable y la importancia de los criptocromos en plantas y animales se ha mostrado relevante y objeto de estudio (vease como ejemplo \cite{cromogeneral})

Para modelar el efecto de un proceso de par radical dependiente del campo en la visión de un animal, es necesario especificar cómo interactúa dicho proceso con la vía visual. A modo de ilustración, suponemos que el proceso de par radicales afecta la sensibilidad de los receptores de luz en el ojo. Esta modulación de sensibilidad dará como resultado un patrón de respuesta que varía en el hemisferio del ojo. Siguiendo las ideas de \cite{visualmagne} llegamos a una resolución como la empleada en la figura \ref{visual}.

\begin{figure}
	\centering
	\includegraphics[width=0.7\textwidth]{pollo_vision}
	\caption{patrones de modulación visual a través del campo geomagnético para un pájaro que mira en diferentes direcciones en diferentes ángulos con el vector de campo magnético. }
	\label{visual}
\end{figure}


\section{Notas finales}
\spacing{1.5}
Estamos ante un campo con mucho recorrido aun por delante. Durante la ultima charla especulamos con la posibilidad de insertar criptocromos en un celula de manera artificial. No solo tenemos que profundizar en este proceso cuantico que ocurre en la naturaleza nos da información sobre la sensibilidad magnetica, si no que, estudiandolo encontramos otras aplicaciones (véase \cite{el2017blue} ).
Aun así, el profesor Ritz señalaba que nos queda mucho por comprender, resaltando que disponemos de herramientas teoricas para el desarrollo, pero aun falta informacion de procesos experimentales para una correcta interpretacion cuantitativa. 

Además durante el congreso pude hablar con el profesor Ritz y transmitirle algunas de mis dudas. 
En primer lugar me parece que el experimento del embudo de Emlen puede mejorarse, sobre todo si tenemos en cuenta el nivel de procesado de imagen actual. Especificamente hablando he trabajo con librerias de reconocimiento visual como OpenCv, basada en Python, la cual puede ofrecer una interesante mejora en precisión y diversificación de los datos de comportamiento, basta ver por ejemplo \ref{animal_1}

\begin{figure}
	\centering
	\includegraphics[width=0.7\textwidth]{animal_track}
	\caption{Ejemplo de uso de la libreria AnimalTrack (ahora en desuso) basada en Opencv}
	\label{animal_1}
\end{figure}

A parte, me resultó bastante curiosa la representación gráfica que nos ofreció. Debido a la falta de experimentos relacionados, y muchas veces (como en el caso de las moscas) a la dificultad de experimentar con animales, propuse la idea de utilizar una red neuronal que aprendiese evolutivamente a ir hacia el norte. Esto es, dada una red neuronal, y dada una entrada como la que se puede ver en \ref{visual}, programar un algoritmo evolutivo que beneficie a los individuos que sean capaces de reconocer estos patrones y orientarse on esa señal visual. Este algoritmo evolutivo premiará aquellas configuraciones de peso y topología de la red que, mediante mutaciones, mejor se estén adaptando (es decir, la idea es utilizar una \textit{ANN evolutiva}). Una vez completado el entrenamiento se pueden realizar experimentos cambiando las condiciones iniciales y, también puede resultar interesante ver que configuraciones son mas sencillas de aprender o el numero minimo de nodos en la red , etc. 
El profesor Ritz se interesó por la idea, y acordamos seguir en contacto para, en el caso de llevarla a cabo proporcionarme todos los datos necesarios para ello.



\section*{Referencias}
\bibliographystyle{apalike}
\bibliography{bibliograf}



\end{document}
