\documentclass[1p]{elsarticle}

\usepackage{lineno,hyperref}
\modulolinenumbers[5]
\usepackage[utf8]{inputenc}
\usepackage[spanish]{babel}
\usepackage{amsmath}
\usepackage{todonotes}
\usepackage{listings}
\usepackage{graphicx}
\usepackage{amsfonts}
\usepackage[toc,page]{appendix}
\usepackage{amssymb}
\newtheorem{thm}{Teorema}
\newtheorem{lem}[thm]{Lema}
\newdefinition{rmk}{Remark}
\newproof{pf}{Demostración}
\newproof{pot}{Demostración del Teorema \ref{thm2}}
 
 \usepackage{listings}
 \usepackage{color}
 
 \definecolor{codegreen}{rgb}{0,0.6,0}
 \definecolor{codegray}{rgb}{0.5,0.5,0.5}
 \definecolor{codepurple}{rgb}{0.58,0,0.82}
 \definecolor{backcolour}{rgb}{0.95,0.95,0.92}
 
 \lstdefinestyle{mystyle}{
 	backgroundcolor=\color{backcolour},   
 	commentstyle=\color{codegreen},
 	keywordstyle=\color{magenta},
 	numberstyle=\tiny\color{white},
 	stringstyle=\color{codepurple},
 	basicstyle=\footnotesize,
 	breakatwhitespace=false,         
 	breaklines=true,                 
 	captionpos=b,                    
 	keepspaces=true,                 
 	numbers=left,                    
 	numbersep=5pt,                  
 	showspaces=false,                
 	showstringspaces=false,
 	showtabs=false,                  
 	tabsize=2
 }
 
 \lstset{style=mystyle}
%%\bibliographystyle{IEEEannot}

%% `Elsevier LaTeX' style
\bibliographystyle{elsarticle-num}
%%%%%%%%%%%%%%%%%%%%%%%
\usepackage{setspace}  
\begin{document}

\begin{frontmatter}

\title{State of art in the Quantum Physics' explanation of the animals' magnetic sense }

%% Group authors per affiliation:
\author{Bartolomé Ortiz Viso}
\address{Master en Física y Matemáticas\\ Universidad de Granada\\23/06/2018}

\begin{abstract}
This work offers a brief look in the quantum physics' explanation of the animals' magnetics sense. It is based on the talks given by Thorsten Ritz in BIOMAT2018 congress, whose main topic was quantum biology. The aim of these pages is to explain the main results in this particular topic (magnetic sensing), its connections with quantum physics and also to offer some other highlights of the talks. Moreover the reader can find some personal opinions and possibles advances that I discussed with Thorsten himself. 
\end{abstract}

\begin{keyword}
 \texttt{Quantum Physics} \sep \texttt{Mathematics}\sep \texttt{Quantum Biology} \sep \texttt{Magnetic Sensing}
\end{keyword}

\end{frontmatter}
\setlength\parindent{0pt}
\linenumbers

\section{Introducción}
\spacing{1.5}


\section{Background biológico}
\spacing{1.5}

Nos encontramos ante un delicado campo de estudio. En esta primera sección vamos a exponer los principales hallazgos en cuanto a el conocimiento de que diversas especies animales poseen la capacidad de sentir campos magnéticos, en particular el campo magnético terrestre.

Destacamos que aun hoy sabemos bastante poco sobre este sentido. Si bien los mecanismos hoy día están investigándose, estamos lejos de comprender este sentido completamente.

Aun asi, aunque no sepamos todos los mecanismos involucrados, conocemos la existencia y algunos de los limites de este sentido gracias a los experimentos que se llevan a cabo con diferentes especies animales. Son las respuesta comportamentales que se observan en los sujetos de los experimentos las que nos presetan el mayor indicio de que este sentido existe y tiene un alto impacto en su dinámica, aun sin saber como funciona exactamente.

En primer lugar, en estos estudios es habitual comprender la tierra como un barra magnetica gigante. Y, nos interesamos por campo magnetico vectorial. Aunque en cada punto podemos encontrar 3 componentes espaciales, solemos medir la componente horizontal, también, es destacable el angulo de inclinación, como veremos durante los experimentos. 



\section{Mecanismo de par radical}
\spacing{1.5}

\section{Diseñando de un sensor óptimo}
\spacing{1.5}



\section{Notas finales}
\spacing{1.5}


\section*{Referencias}
\spacing{1.5}
\bibliography{bibliograf}

\begin{appendices}
	

\end{appendices}


\end{document}
