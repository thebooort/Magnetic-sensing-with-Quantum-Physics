\documentclass[1p]{elsarticle}

\usepackage{lineno,hyperref}
\modulolinenumbers[5]
\usepackage[utf8]{inputenc}
\usepackage[spanish]{babel}
\usepackage{amsmath}
\usepackage{todonotes}
\usepackage{listings}
\usepackage{graphicx}
\usepackage{amsfonts}
\usepackage{wrapfig}
\usepackage[toc,page]{appendix}
\usepackage{amssymb}
\newtheorem{thm}{Teorema}
\newtheorem{lem}[thm]{Lema}
\newdefinition{rmk}{Remark}
\newproof{pf}{Demostración}
\newproof{pot}{Demostración del Teorema \ref{thm2}}
 \graphicspath{ {./images/} }
 \usepackage{listings}
 \usepackage{color}
 
 \definecolor{codegreen}{rgb}{0,0.6,0}
 \definecolor{codegray}{rgb}{0.5,0.5,0.5}
 \definecolor{codepurple}{rgb}{0.58,0,0.82}
 \definecolor{backcolour}{rgb}{0.95,0.95,0.92}
 
 \lstdefinestyle{mystyle}{
 	backgroundcolor=\color{backcolour},   
 	commentstyle=\color{codegreen},
 	keywordstyle=\color{magenta},
 	numberstyle=\tiny\color{white},
 	stringstyle=\color{codepurple},
 	basicstyle=\footnotesize,
 	breakatwhitespace=false,         
 	breaklines=true,                 
 	captionpos=b,                    
 	keepspaces=true,                 
 	numbers=left,                    
 	numbersep=5pt,                  
 	showspaces=false,                
 	showstringspaces=false,
 	showtabs=false,                  
 	tabsize=2
 }
 
 \lstset{style=mystyle}
%%\bibliographystyle{IEEEannot}

%% `Elsevier LaTeX' style
\bibliographystyle{elsarticle-num}
%%%%%%%%%%%%%%%%%%%%%%%
\usepackage{setspace}  
\begin{document}

\begin{frontmatter}

\title{State of art in the Quantum Physics' explanation of the animals' magnetic sense }

%% Group authors per affiliation:
\author{Bartolomé Ortiz Viso}
\address{Master en Física y Matemáticas\\ Universidad de Granada\\23/06/2018}

\begin{abstract}
This work offers a brief look in the quantum physics' explanation of the animals' magnetics sense. It is based on the talks given by Thorsten Ritz in BIOMAT2018 congress, whose main topic was quantum biology. The aim of these pages is to explain the main results in this particular topic (magnetic sensing), its connections with quantum physics and also to offer some other highlights of the talks. Moreover the reader can find some personal opinions and possibles advances that I discussed with Thorsten himself. 
\end{abstract}

\begin{keyword}
 \texttt{Quantum Physics} \sep \texttt{Mathematics}\sep \texttt{Quantum Biology} \sep \texttt{Magnetic Sensing}
\end{keyword}

\end{frontmatter}
\setlength\parindent{0pt}
\linenumbers

\section{Introducción}
\spacing{1.5}


\section{Background biológico}
\spacing{1.5}

Nos encontramos ante un delicado campo de estudio. En esta primera sección vamos a exponer los principales hallazgos en cuanto a el conocimiento de que diversas especies animales poseen la capacidad de sentir campos magnéticos, en particular el campo magnético terrestre.

Destacamos que aun hoy sabemos bastante poco sobre este sentido. Si bien los mecanismos hoy día están investigándose, estamos lejos de comprender este sentido completamente. Factores como los mecanismos fisicos implicados, las moleculas receptoras, la transducción de la señal o el procesado neuronal de la misma, son aun objetivo d intenso debate y estudio.

Aun asi, aunque no sepamos todos los mecanismos involucrados, conocemos la existencia y algunos de los límites de este sentido, gracias a los experimentos que se llevan a cabo con diferentes especies animales. Son las respuesta comportamentales que se observan en los sujetos de los experimentos, las que nos presentan el mayor indicio de que este sentido existe y tiene un alto impacto en su dinámica, aun sin saber como funciona exactamente.

En primer lugar, en estos estudios es habitual comprender la tierra como un barra magnética gigante (gigante no nos debe conducir a error: el campo magnético terrestre es difícil de detectar biológicamente). Y, nos interesamos por campo magnético vectorial. Aunque en cada punto podemos encontrar 2 componentes : horizontal y vertical, solemos medir la componente horizontal, y también es destacable el ángulo de inclinación, como veremos durante los experimentos. 

Uno de los animales más habituales en este tipo de experimentos son los pájaros. Es de sobra conocido que muchas aves tienen pautas de migración muy interesantes y complejas, en las que tener sensibilidad al campo magnético terrestre juega un papel crucial.

 En este área los primeros experimentos fueron gracias al desarrollo de instrumental experimental especifico, puesto que los métodos observacionales se mostraron ineficaces. Como se puede leer en \cite{embudo}, los investigadores desarrollaron un embudo \ref{embudo_1} para percibir la dirección que toman los pájaros durante los experimentos. 
  \begin{figure}
  	\centering
  	\includegraphics[width=0.5\textwidth]{emlen_funnel}
  	\caption{Diagrama de embudo de Emlen}
  	\label{embudo_1}
  \end{figure}
  
 Uno de los primeros estudios con esta técnica centrado en el campo magnético terrestre fue llevado a cabo en 1972 \cite{petirrojo}. En el se escogió al petirrojo europeo (\textit{Erithacus rubecula}),el cual está distribuido por toda Europa, principalmente en la región meridional y occidental del continente, donde habita todo el año, siendo migrante parcial en el norte de Europa y noroeste de África.
Se recogieron varios de estos individuos y, una vez dentro de un embudo de Emlen, se procedió a cambiar artificialmente el campo magnético. Se pudieron obtener algunas comclusiones: 
\begin{itemize}
	\item Los pájaros siguieron el norte, tanto en el grupo de control como cuando el norte era alterado artificialmente.\ref{embudo_2}
	\item Los pájaros eran capaces de detectar la inclinación, pero no la polaridad, por lo que la componente vertical es importante.
\end{itemize}
\begin{figure}
	\centering
	\includegraphics[width=0.7\textwidth]{embudopollos}
	\caption{Resultados del cambio en el campo magnetico}
	\label{embudo_2}
\end{figure}

Se compuso así la primera prueba de la sensibilidad magnetica de estos animales. Más tarde fueron incluidos multitud de experimentos relacionados con esta capacidad, los cuales voy a nombrar brevemente para finalizar esta parte.

\begin{itemize}
	\item Se ha observado ciertas relaciones entre la luz y la cantidad de la misma. Aunque no hay una correlacion obvia entre fotoreceptores y comportamiento, sin embargo, la manipulacion con el ciclo regulatorio puede ser la causa. Volveremos sobre este tema durante las evidencia de la importancia de este 
	\item Se ha observado que tapar el ojo derecho produce desorientación, mientras que tapar el ojo izquierdo no, con lo que se plantea la posibilidad de que la "brujula" se encuentre en este ojo.
	\item Se han encontrado evidencias de que los pajaros pueden resetear su percepcion en ciclos de un día. Estudios mostraban que pajaros que tenían una dirección equivocada de manera artificial, corregían su rumbo cuando pasaba un día.
\end{itemize}

Por último, estudios de comportamientos se han llevado a cabo en muchos otros animales, de los cuales, destacamos: 
\begin{itemize}
	\item Gallina común: En este caso el estudio presenta una clara muestra de condicionamiento magnético
	\item Tortuga boba: este animal ofrece unos comportamientos migratorios fascinantes. En experimentos llevados a cabo en piscinas, se pudo observar esa percepcion al campo magnetico como la mostrada en pájaros.
	\item Las moscas de la fruta tambien presentan este tipo de comportamiento
	\item Muchos otros animales: invertebrados, mamiferos, peces, etc.
\end{itemize}

Hasta la fecha no se ha encontrado evidencia alguna de que los seres humanos poseamos esta capacidad.

Una vez que tenemos una visión global de este fenómeno, vamos a adentrarnos en los mecanismos que pueden motivarlo.


\section{Mecanismo de par radical}
\spacing{1.5}

\section{Diseñando de un sensor óptimo}
\spacing{1.5}



\section{Notas finales}
\spacing{1.5}


\section*{Referencias}
\spacing{1.5}
\bibliographystyle{apalike}
\bibliography{bibliograf}



\end{document}
